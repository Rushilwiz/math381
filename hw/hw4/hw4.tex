\documentclass{article}

\usepackage{fancyhdr}
\usepackage{extramarks}
\usepackage{amsmath}
\usepackage{amsthm}
\usepackage{amsfonts}
\usepackage{tikz}
\usepackage[plain]{algorithm}
\usepackage{algpseudocode}
\usepackage[shortlabels]{enumitem}
\usepackage{mathtools}
\usepackage{amssymb}

\usetikzlibrary{automata,positioning}

%
% Basic Document Settings
%

\topmargin=-0.45in
\evensidemargin=0in
\oddsidemargin=0in
\textwidth=6.5in
\textheight=9.0in
\headsep=0.25in

\linespread{1.1}

\pagestyle{fancy}
\lhead{\hmwkAuthorName}
\chead{\hmwkClass\ (\hmwkClassInstructor\ \hmwkClassTime): \hmwkTitle}
\lfoot{\lastxmark}
\cfoot{\thepage}

\renewcommand\headrulewidth{0.4pt}
\renewcommand\footrulewidth{0.4pt}

\setlength\parindent{0pt}

%
% Create Problem Sections
%

\newcommand{\enterProblemHeader}[1]{
    \nobreak\extramarks{}{Problem \arabic{#1} continued on next page\ldots}\nobreak{}
    \nobreak\extramarks{Problem \arabic{#1} (continued)}{Problem \arabic{#1} continued on next page\ldots}\nobreak{}
}

\newcommand{\exitProblemHeader}[1]{
    \nobreak\extramarks{Problem \arabic{#1} (continued)}{Problem \arabic{#1} continued on next page\ldots}\nobreak{}
    \stepcounter{#1}
    \nobreak\extramarks{Problem \arabic{#1}}{}\nobreak{}
}

\setcounter{secnumdepth}{0}
\newcounter{partCounter}
\newcounter{homeworkProblemCounter}
\setcounter{homeworkProblemCounter}{1}
\nobreak\extramarks{Problem \arabic{homeworkProblemCounter}}{}\nobreak{}

%
% Homework Problem Environment
%
% This environment takes an optional argument. When given, it will adjust the
% problem counter. This is useful for when the problems given for your
% assignment aren't sequential. See the last 3 problems of this template for an
% example.
%
\newenvironment{homeworkProblem}[1][-1]{
    \ifnum#1>0
        \setcounter{homeworkProblemCounter}{#1}
    \fi
    \section{Problem \arabic{homeworkProblemCounter}}
    \setcounter{partCounter}{1}
    \enterProblemHeader{homeworkProblemCounter}
}{
    \exitProblemHeader{homeworkProblemCounter}
}

\newcommand{\hmwkTitle}{Homework 4}
\newcommand{\hmwkDueDate}{October 5, 2023}
\newcommand{\hmwkClass}{Discrete Math}
\newcommand{\hmwkClassTime}{Section 003}
\newcommand{\hmwkClassInstructor}{Reese Lance}
\newcommand{\hmwkAuthorName}{\textbf{Rushil Umaretiya}}

%
% Title Page
%

\title{
    \vspace{2in}
    \textmd{\textbf{\hmwkClass:\ \hmwkTitle}}\\
    \normalsize\vspace{0.1in}\small{Tuesday/Thursday 11:00-12:15, Phillips 383}\\
    \vspace{0.1in}\large{\textit{\hmwkClassInstructor\ - \hmwkClassTime}}
    \vspace{3in}
}

\author{\hmwkAuthorName\\\small{rumareti@unc.edu}}
\date{}

\renewcommand{\part}[1]{\textbf{\large Part \Alph{partCounter}}\stepcounter{partCounter}\\}

%
% Various Helper Commands
%

% Useful for algorithms
\newcommand{\alg}[1]{\textsc{\bfseries \footnotesize #1}}

% For derivatives
\newcommand{\deriv}[1]{\frac{\mathrm{d}}{\mathrm{d}x} (#1)}

% For partial derivatives
\newcommand{\pderiv}[2]{\frac{\partial}{\partial #1} (#2)}

% Integral dx
\newcommand{\dx}{\mathrm{d}x}

% Alias for the Solution section header
\newcommand{\solution}{\textbf{\large Solution}}

\newcommand{\unit}[1]{\section{Unit #1}}
\newcommand{\problem}[1]{\textbf{\##1}}
\newcommand{\prob}[1]{\problem{#1}}


% Probability commands: Expectation, Variance, Covariance, Bias
\newcommand{\E}{\mathrm{E}}
\newcommand{\Var}{\mathrm{Var}}
\newcommand{\Cov}{\mathrm{Cov}}
\newcommand{\Bias}{\mathrm{Bias}}

\renewcommand{\And}{\wedge}
\newcommand{\Or}{\vee}
\newcommand{\Xor}{\oplus}
\newcommand{\Not}{\neg}
\newcommand{\Implies}{\rightarrow}
\newcommand{\Iff}{\leftrightarrow}

\newcommand{\AllIntegers}{\mathbb{Z}}
\newcommand{\AllRationals}{\mathbb{Q}}
\newcommand{\AllReals}{\mathbb{R}}
\newcommand{\AllComplexes}{\mathbb{C}}

\begin{document}

\maketitle

\pagebreak

\unit{5.1}
\prob{10}
\begin{enumerate}[a)]
    \item Find a formula for
    \begin{align*}
        \frac{1}{1\cdot2}+\frac{1}{2\cdot3}+\ldots+\frac{1}{n(n+1)}
    \end{align*}
    by examining the values of this expression for small values of \(n\).\\

Given the following equation,
\begin{align*}
    P(n) &= \sum_{i=1}^{n}\frac{1}{n(n+1)}\\
\end{align*}
We can find the following values for \(P(x)\) and generalize a formula for \(P(n)\).
\begin{align*}
    P(1) &&= \frac{1}{2}\\
    P(2) &= \frac{1}{2}+\frac{1}{6}&=\frac{2}{3}\\
    P(3) &= \frac{1}{2}+\frac{1}{6}+\frac{1}{12}&=\frac{3}{4}\\
    P(4) &= \frac{1}{2}+\frac{1}{6}+\frac{1}{12}+\frac{1}{20}&=\frac{4}{5}\\
    P(n) &&= \frac{n}{n+1}
\end{align*}
    \item Prove the formula you conjectured in part (a).
\begin{proof}
    We will prove the formula by induction.\\
    \textbf{Base Case:} \(n=1\)
    \begin{align*}
        P(1) = \frac{1}{1+1} = \frac{1}{2}
    \end{align*}
    \textbf{Inductive step:}
    Given P(n), we will prove P(n+1).
    \begin{align*}
        P(n+1) &= \sum_{i=1}^{n+1}\frac{1}{n(n+1)}\\
        &= \sum_{i=1}^{n}\frac{1}{n(n+1)}+\frac{1}{(n+1)(n+2)}\\
        &= \frac{n}{n+1}+\frac{1}{(n+1)(n+2)}\\
        &= \frac{n(n+2)+1}{(n+1)(n+2)}\\
        &= \frac{(n+1)^2}{(n+1)(n+2)}\\
        &= \frac{n+1}{n+2}
    \end{align*}
\end{proof}
\end{enumerate}
\pagebreak
\prob{34}
Prove that 6 divides \(n^3 - n\) whenever \(n\) is a nonnegative integer.\\
\begin{proof}
    Given \(P(n) = \exists k \in \AllIntegers, n^3-n = 6k\), we will show that \(\forall n \in \AllIntegers^+(P(n))\) by induction.\\
    \textbf{Base Case:} \(n=0\)
    \begin{align*}
        0^3 - 0 = 6(0)
    \end{align*}
    \textbf{Inductive step:}
    Given P(n), we will prove P(n+1).
    \begin{align*}(n+1)
        (n+1)^3-(n+1) &= n^3+3n^2+3n+1-(n+1)\\
        &= n^3+3n^2+2n\\
        &= n^3-n+(3n^2+3n)\\
        &= (n^3 - n) + 3(n)(n+1)
    \end{align*}
    Since we are assuming \(P(n)\), we can affirm that \(n^3-n\) is true. Now we can show that 6 also divides the second term,
    \begin{align*}
        \exists k \in \AllIntegers, 3(n)(n+1) &= 6k\\
        n(n+1) &= 2k
    \end{align*}
    Since any odd and even integer multiplied together is even, we can affirm that 6 divides \(3(n)(n+1)\). Since we know that,
    \begin{align*}
        \forall a,b,n \in \AllIntegers,\\n|a, n|b \implies n|(a+b).
    \end{align*}
    6 must divide \((n^3 - n) + 3(n)(n+1)\). Thus \(P(n+1)\) is true.
\end{proof}

\pagebreak
\prob{64}
Use mathematical induction to prove that if \(p\) is a prime and \(p | a_1a_2 \cdots a_n\), where \(a_i\) is an integer for \(i = 1, 2, 3, \dots , n\), then \(p | a_i\) for some integer \(i\). 
\begin{align*}
    P(n) = \forall a_1, a_2, \dots, a_n \in \AllIntegers, p | a_1a_2 \cdots a_n \implies p | a_i \text{ for some } i \in \AllIntegers .\\
\end{align*}
\textbf{Base Case:} \(n=1\)
\begin{align*}
    p | a_1 \implies p | a_1
\end{align*}
\textbf{Inductive Step:} Given P(n), we will prove P(n+1).
\begin{align*}
    P(n+1) &= p|a_1a_2\cdots a_na_{n+1} \implies p|a_i \text{ for some } i \in \AllIntegers .\\
\end{align*}
Since we know that \(p\) is prime,
\begin{align*}
    p|a_1a_2\cdots a_na_{n+1} &\implies p|a_1a_2\cdots a_n \text{ or } p|a_{n+1}\\
    &\implies P(n) \text{ or } p|a_{n+1}\\
    &\implies p|a_i \text{ for some } i \in \AllIntegers \text{ or } p|a_{n+1}\\
    &\implies P(n+1)
\end{align*}
We have shown that \(P(n) \implies P(n+1)\), thus \(P(n+1)\) is true.
\pagebreak
\unit{2.1} 
\prob{2} Use set builder notation to give a description of each of these sets.
\begin{enumerate}[a)]
    \item \{{0, 3, 6, 9, 12}\}\\
    \begin{align*}
        A = \{x \in \AllIntegers^+ | x = 3n, 0 \leq n \leq 4\}
    \end{align*}
    \item \{{-3,-2,-1, 0, 1, 2, 3}\}
    \begin{align*}
        B = \{x \in \AllIntegers | -3 \leq x \leq 3\}
    \end{align*}
    \item \(\{m,n,o,p\}\)
    \begin{align*}
        C = \{x | x \text{ is a lowercase letter in the alphabet from m to p}\}
    \end{align*}
\end{enumerate}
\pagebreak
\prob{6} For each of these pairs of sets, determine whether the first is a subset of the second, the second is a subset of the first, or neither is a subset of the other.
\begin{enumerate}[a)]
    \item the set of people who speak English, the set of people who speak English with an Australian accent
    \item the set of fruits, the set of citrus fruits
    \item the set of students studying discrete mathematics, the set of students studying data structures
\end{enumerate}
\pagebreak
\prob{12} Determine whether these statements are true or false.
\begin{enumerate}[a)]
    \item \(\emptyset \in \{\emptyset\}\)
    \item \(\emptyset \in \{\emptyset, \{\emptyset\}\}\)
    \item \(\{\emptyset\} \in \{\emptyset\}\)
    \item \(\{\emptyset\} \in \{\{\emptyset\}\}\)
    \item \(\{\emptyset\} \subset \{\emptyset, \{\emptyset\}\}\)
    \item \(\{\{\emptyset\}\} \subset \{\emptyset, \{\emptyset\}\}\)
    \item \(\{\{\emptyset\}\} \subset \{\{\emptyset\}, \{\emptyset\}\}\)
\end{enumerate}
\pagebreak
\prob{18}
Use a Venn diagram to illustrate the relationships \(A \subset B\) and \(A \subset C\).

\pagebreak
\prob{22}
What is the cardinality of each of these sets?
\begin{enumerate}[a)]
    \item \(\emptyset\)
    \item \(\{\emptyset\}\)
    \item \(\{\emptyset, \{\emptyset\}\}\)
    \item \(\{\emptyset, \{\emptyset\}, \{\emptyset, \{\emptyset\}\}\}\)
\end{enumerate}

\pagebreak
\prob{32} Suppose that \(A \times B = \emptyset\), where \(A \text{ and } B\) are sets. What can you conclude?

\pagebreak
\prob{44} Prove or disprove that if \(A, B, \text{and } C\) are nonempty sets and \(A \times B = A \times C\text{, then } B = C\).

\end{document}